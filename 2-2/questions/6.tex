\section{\lr{Max Flow}}
در این سوال از کد‌های
\link{https://www.geeksforgeeks.org/dinics-algorithm-maximum-flow/}{این لینک}
و
\link{https://github.com/KatiaVi/418MaxFloProject}{این لینک}
استفاده کردم. صرفا فقط مقداری کد
\lr{Geeks For Geeks}
را عوض کردم که از
\lr{STL}های
خود
\lr{CPP}
استفاده کند. صرفا جایی که می‌توان الگوریتم را بهینه کرد جایی است که از
\lr{BFS}
استفاده می‌کنیم. می‌توان همزمان تمامی همسایه‌ها را باز کرد و دید. برای این کار تابع
\lr{bfs}
را صرفا به کمک
\codeword{\#pragma omp parallel for}
موازی سازی می‌کنیم. دقت کنید که اضافه کردن هر راس به
\lr{queue}
نیز نیازمند
\lr{lock}
است. حال سرعت برنامه را تست می‌کنیم.
\samplebox{./serial.out delaunay\_n17.txt\\
Creating graph with size of 131072\\
Read graph file\\
Maximum flow 1890\\
Took 1.07326 s}
\samplebox{./parallel.out delaunay\_n17.txt\\
Creating graph with size of 131072\\
Read graph file\\
Maximum flow 1890\\
Took 13.2472 s}
این افزایش سرعت احتمالا از این موضوع می‌آید که درجات گراف ما کم است و در نتیجه
\lr{BFS}
خیلی سودی ندارد. برای چک کردن این موضوع برنامه‌ای نوشتم که تعداد راس‌ها با درجات مختلف را به من گزارش
می‌داد. نتیجه به صورت زیر است:
\begin{figure}[H]
    \centering
    \begin{tikzpicture}
    \begin{axis}
    \addplot table [x=degree, y=count, col sep=comma] {codes/max-flow/n17.csv};
    \end{axis}
    \end{tikzpicture}
    \caption{تعداد راس‌ها بر حسب درجه}
\end{figure}
همان طور که مشخص است اکثر راس‌ها درجات کمی دارند. برای همین اصلا نمی‌صرفد که موازی اجرا کنیم.
حال کمی کد را تغییر می‌دهیم که فقط راس‌ها با درجات بیشتر از 16 موازی سازی شوند.
\samplebox{./parallel2.out delaunay\_n17.txt\\
Creating graph with size of 131072\\
Read graph file\\
Maximum flow 1890\\
Took 1.26311 s}
همان طور که مشاهده می‌شود سرعت برنامه بهتر شد ولی باز‌هم خیلی خوب نیست. به همین جهت من یک تست دیگر
ساختم که یک گراف کامل است. تست را با 2048 راس شروع کردم و نتیجه سری و موازی به صورت زیر شد:
\samplebox{./serial.out test.txt \\
Creating graph with size of 2048\\
Read graph file\\
Maximum flow 997610\\
Took 0.0543808 s}
\samplebox{./parallel2.out test.txt\\
Creating graph with size of 2048\\
Read graph file\\
Maximum flow 997610\\
Took 0.0735186 s}
همان طور که مشاهده می‌شود باز هم سرعت برنامه کمتر شد. به نظر من دلیل اصلی آن این است که من خوب
موازی سازی ‌نمی‌کنم. مثلا در لینکی که در اول کار آورده بودم یک قسمت دیگر از برنامه نیز موازی سازی
شده بود.