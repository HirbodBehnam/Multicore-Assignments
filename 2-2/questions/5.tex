\section{\lr{Merge sort}}
در ابتدا کد یک نفر برای
\lr{merge sort}
را از اینترنت می‌دزدیم. من از
\link{https://www.programiz.com/dsa/merge-sort}{این سایت}
استفاده کردم. در ابتدا سرعت برنامه به صورت سری را اندازه گیری می‌کنیم. برای من سرعت برنامه
$0.0833 \text{ms}$
شد. سپس صرفا به کمک
\codeword{omp parallel sections}
دو تابع
\codeword{mergeSort}
را به صورت موازی اجرا می‌کنیم. سپس برنامه را دوباره
\lr{benchmark}
می‌کنیم. این بار سرعت برنامه برابر
$0.516324 \text{ms}$
می‌شود! این موضوع به دلیل این است که سربار ساخت و همگام سازی ترد‌ها در مراحل پایانی
\lr{merge sort}
زیادتر از خود عملیات
\lr{sort}
است. همچنین مانند سوال ممکن است که در مراحل پایانی
\lr{false sharing}
رخ دهد. پس برای حل کردن این مشکل کافی است که قسمت‌های آخری
\lr{merge sort}
را به صورت سری انجام دهیم. به عبارت دیگر در صورتی که تفاوت
$l$ و $r$
کمتر از عددی بود دیگر به صورت سری انجام دهیم عملیات را به جای موازی.
برای همین کد قسمت
\lr{merge sort}
به صورت زیر می‌شود:
\codesample{codes/merge-sort/part.c1}
حال سعی می‌کنیم که یک
\lr{threshold}
مناسب برای موازی سازی پیدا کنیم. بدین جهت با
\lr{threshold}های
مختلف برنامه را اجرا می‌کنیم. نتیجه به صورت زیر است:
\begin{figure}[H]
    \centering
    \begin{tikzpicture}
    \begin{axis}
    \addplot table [x=threshold, y=time, col sep=comma] {codes/merge-sort/threshold.csv};
    \end{axis}
    \end{tikzpicture}
    \caption{سرعت اجرای برنامه بر حسب $\log_2 (threshold)$}
\end{figure}
همان طور که مشاهده می‌شود بعد از حدود
$256$
به عنوان
\lr{threshold}
نرخ نمودار ثابت می‌شود و بعد از مدتی صعودی می‌شود. پس بهتر است که
\lr{threshold}
را برابر
$256$
قرار دهیم. با این کار سرعت برنامه به حدود
$0.05159 \text{ms}$
می‌رسد.