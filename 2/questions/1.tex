\section{\lr{Branch Prediction}}
% https://courses.cs.washington.edu/courses/csep548/06au/lectures/branchPred.pdf
طبق صحبت‌هایی که با فرود و آرش داشتم طوری که این سیستم کار می‌کند بدین صورت است که
\lr{GHR}
یک
\lr{shift register}
است. هر بار یکی عدد را به چپ شیفت می‌دیم و
\lr{taken} یا \lr{not taken}
بودن
\lr{branch}
را وارد می‌کنیم. سپس به نگاه کردن به آن رجیستر و در آوردن
\lr{entry}
مربوط به آن در جدول
\lr{PHT}
حدس می‌زنیم که یک
\lr{branch}،
\lr{taken}
است یا خیر.

می‌توان مانند زیر صرفا حالت
\lr{GHR}
را نوشت در کنار هر
\lr{branch}
که باید یا نباید
\lr{predict}
شود.
\begin{latin}
\centering
\begin{tabular}{|c|c|c|}
    \hline
    Branch \# & GHR & Taken\\
    \hline
    0 & NN & T\\
    \hline
    1 & NT & T\\
    \hline
    2 & TT & T\\
    \hline
    3 & TT & N\\
    \hline
    4 & TN & T\\
    \hline
    5 & NT & N\\
    \hline
    6 & TN & T\\
    \hline
    7 & NT & T\\
    \hline
    8 & TT & T\\
    \hline
    9 & TT & N\\
    \hline
\end{tabular}
\end{latin}
هر کدام از این حالات
\lr{GHR}
را بررسی می‌کنیم.
\begin{itemize}
    \item \lr{TT}:‌ در این حالت در ابتدا یک
    \lr{branch}
    که باید
    $T$
    باشد باید
    $N$
    \lr{predict}
    شود. سپس باید
    $T$
    باید
    \lr{predict}
    شود و سپس
    دوباره
    $N$ و سپس $T$
    شود. پس تنها حالت اولیه ممکن برای این حالت
    $01$
    است.
    \item \lr{TN}: در این حالت دو برنچ باید به صورت
    $N$
    پیش بینی شوند. پس باید مقدار اولیه
    $00$
    باشد.
    \item \lr{NT}: مثل \lr{TT} باشد باشد.
    \item \lr{NN}: فقط یک برنچ باید \lr{not taken} \lr{predict} شود پس باید یا
    $00$ باشد یا $01$.
\end{itemize}

برای قسمت دوم سوال این بار برای هر یک از
\lr{branch}ها
یک رجیستر جداگانه دارند. برای این کار کافی است که برای
\lr{branch}
اول جدول به صورت زیر باشد:
\begin{latin}
    \centering
    \begin{tabular}{|c|c|}
        \hline
        PHT Entry & Value\\
        \hline
        NN & 11\\
        \hline
        NT & 11\\
        \hline
        TN & Dont care\\
        \hline
        TT & 11\\
        \hline
    \end{tabular}
\end{latin}
برای برنچ دوم عملا رشته‌ی زیر
\lr{branch}های
ما هستند:
\begin{gather*}
    TNNTNNTNN\dots
\end{gather*}
برای همین جدول ما باید به صورت زیر باشد:
\begin{latin}
    \centering
    \begin{tabular}{|c|c|}
        \hline
        PHT Entry & Value\\
        \hline
        NN & 11\\
        \hline
        NT & 00\\
        \hline
        TN & 00\\
        \hline
        TT & Dont care\\
        \hline
    \end{tabular}
\end{latin}
با ادغام دو جدول مشکلی بر روی
\lr{NN}، \lr{TN} و \lr{TT}
نداریم. ولی بر روی
\lr{NT}
مشکل داریم. با خواندن دقیق تر سوال متوجه می‌شویم که سوال صرفا از ما می‌خواهد که الگوی کلی
$TTTNTN\dots$
 بررسی کنیم در حالت \lr{stable} که در اینجا اصلا برای
\lr{branch}
اول اصلا
$NN$ و $NT$
هم مهم نیست. پس در نتیجه جدول نهایی ما برابر می‌شود با:
\begin{latin}
    \centering
    \begin{tabular}{|c|c|}
        \hline
        PHT Entry & Value\\
        \hline
        NN & 11\\
        \hline
        NT & 00\\
        \hline
        TN & 00\\
        \hline
        TT & 11\\
        \hline
    \end{tabular}
\end{latin}
