\section{\lr{Cache Coherence (II)}}
در ابتدا متوجه می‌شویم که از آنجا که هیچ دستوری بر روی
$P_3$
اجرا نمی‌شود پس می‌توان نتیچه گرفت که
\lr{tag}های
ما دست نخورده‌اند. پس نهایتا
\lr{MESI}
آن عوض شده است. با توجه بیشتر متوجه می‌شویم که به آدرس‌های این فیلد‌ها نیز دسترسی نداریم! پس حالت کش
$P_3$
دقیقا همان است که در ابتدا بوده.

حال به سراغ
$P_2$
می‌رویم. در این هسته، تنها یک
\lr{load}
از آدرس
\lr{0x1FFFFF40}
اتفاق می‌افتد و این بلاک مموری در شماره
\lr{set 1}
می‌رود. حال نکته‌ای که وجود دارد این است که بقیه‌ی
\lr{set}ها
که دست نخورده باقی باید مانده باشند از ابتدا. همچنین دقت کنید که کس دیگری به بلاک‌های موجود در
$P_2$
دسترسی نداشته است. اما برای
\lr{set 2}
چندین حالت می‌توان در نظر گرفت.
