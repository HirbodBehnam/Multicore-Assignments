\section{SIMD}
% https://safari.ethz.ch/architecture/fall2017/lib/exe/fetch.php?media=main-sol-midterm.pdf
\begin{enumerate}
    \item به نظر من \lr{vector processor}ها فضای کمتری درگیر می‌کنند
    چرا که برخلاف \lr{array processor} نیازی نیست که
    به تعداد خانه‌های داخل
    \lr{register}
    به عنوان مثال
    \lr{adder} یا ضرب کننده
    داشته باشیم و صرفا یکی کافی است.
    \item اگر فرض کنیم که هر یک از عملیات به صورت
    \lr{pipelined}
    به خود
    \lr{PU}
    داده می‌شوند آنگاه کافی است که معادله‌ی زیر را حل کنیم:
    \begin{gather*}
        5 + \text{\lr{VLEN}} - 1 + 5 + \text{\lr{VLEN}} - 1 + 15 + \text{\lr{VLEN}} - 1 = 52\\
        \implies 3 \text{\lr{VLEN}} = 30 \implies \text{\lr{VLEN}} = 10
    \end{gather*}
    حال اگر یک
    \lr{array processor}
    داشتیم دیگر نیازی به
    \lr{pipeline}
    کردن نبود و برای تمامی خانه‌ها
    \lr{PU}
    جداگانه وجود داشت.
    \begin{gather*}
        5 + 5 + 15 = 25
    \end{gather*}
    \item با توجه به بند سوم نمی‌توان \lr{chain} انجام داد پس کافی است که معادله‌ی زیر را حل کنیم.
    در این معادله در ابتدا فرض می‌کنیم که تعداد المان‌های بردار کمتر مساوی 16 است. نکته‌ای که در این سوال وجود
    دارد این است که در دو دستور
    \lr{load}
    اول منتظر نمی‌مانیم که کامل داده در مموری لود شود. بلکه به محض اینکه بانک اول آزاد شد، بردار دوم را
    درخواست می‌دهیم.
    \begin{gather*}
        20 + 20 + \text{\lr{VLEN}} - 1 + 5 + \text{\lr{VLEN}} - 1 + 1 + \text{\lr{VLEN}} - 1 + 20 + \text{\lr{VLEN}} - 1 = 94\\
        \implies 4 \text{\lr{VLEN}} + 62 = 94 \implies \text{\lr{VLEN}} = 8
    \end{gather*}
    \item در ابتدا تعداد عملیات مورد نیاز برای
    \lr{VADD} و \lr{VSHR}
    را از تعداد کلاک‌ها کم می‌کنیم:
    \begin{gather*}
        170 - (5 + 16 - 1) - (1 + 16 - 1) = 134
    \end{gather*}
    اگر تعداد بانک‌ها به اندازه‌ی کافی بود
    $20 + 20 + 15 + 20 + 15 = 90$
    کلاک عملیات انجام دهیم. اما مشاهده می‌شود که کلاک‌های بدست آمده کمتر از 134 است. پس تعداد بانک‌ها از 16 کمتر است.
    حال چک می‌کنیم که آیا به قدری است که نیاز باشد برای هر
    \lr{load} و \lr{store}
    دوبار بانک‌ها را دور بزنیم. برای این کار عبارت زیر را حساب می‌کنیم:
    \begin{gather*}
        20 + 20 + 20 + 20 + x + 20 + 20 + x = \implies 120 + 2x = 134 \implies x = 7
    \end{gather*}
    اینکه
    $x = 7$
    شد نشان می‌دهد که بعد از لود کردن اولین خانه، 7 خانه‌ی دیگر باید لود شوند. این نشان می‌دهد که تعداد بانک‌های
    ما 8 است.
\end{enumerate}